\documentclass{article}

\usepackage{struggle}

% 

\begin{document}

\subsection*{Torus}
Designate a piece to use this power (traditionally blue).
This piece treats the board as if it wraps around horizontally.
That is, if the Torus piece is on the edge, it can move to the opposite edge of the same row, and can jump similarly.
Further, when this piece encounters an edge moving forward, it may move forward as if it were in the opposite spot on its row (illustrated below).

\

\begin{center}
\begin{struggleboard}
  \piece{i1}{blue};
  \highlight{i2}{purple};
  \highlight{h2}{purple};
  \highlight{h1}{purple};
  \highlight{i5}{purple};
  % \highlight{i2}{purple};
  % \highlight{i2}{purple};
\end{struggleboard}
\begin{struggleboard}
  \piece{d1}{blue};
  \piece{d4}{yellow};
  \highlight{c1}{purple};
  \highlight{d3}{purple};
  \highlight{c3}{purple};
\end{struggleboard}
\begin{struggleboard}
  \piece{h1}{blue};
  \highlight{g1}{purple};
  \highlight{g5}{purple};
  \highlight{h6}{purple};
  \highlight{h2}{purple};
\end{struggleboard}

\

\

\begin{struggleboard}{0.5}
  \piece{e5}{blue};
  \piece{d1}{green};
  \highlight{e1}{purple};
  \highlight{e4}{purple};
  \highlight{d4}{purple};
  \highlight{c1}{purple};
\end{struggleboard}
\quad
\begin{struggleboard}
  \piece{g5}{blue};
  \piece{f2}{black};
  \highlight{f1}{purple};
  \highlight{e2}{purple};
  \highlight{g1}{purple};
  \highlight{g4}{purple};
  \highlight{f5}{purple};
  \highlight{f6}{purple};
\end{struggleboard}
\end{center}

\newpage

\subsection*{Caboose}

If three or more of your nonwhite pieces line up in a row,
you can use your turn to advance all the pieces one step in the direction of the line.

\

\begin{center}
\begin{struggleboard}
  \piece{h3}{red};
  \piece{g3}{black};
  \piece{f4}{blue};
  \move{f4}{e4};
  \move{g3}{f4};
  \move{h3}{g3};
\end{struggleboard}
\begin{struggleboard}
  \piece{i1}{red};
  \piece{i2}{yellow};
  \piece{i3}{green};
  \move{i3}{i4};
  \move{i2}{i3};
  \move{i1}{i2};
\end{struggleboard}
\begin{struggleboard}
  \piece{h5}{red};
  \piece{g4}{green};
  \piece{f4}{blue};
  \piece{e3}{yellow};
  \move{e3}{d2};
  \move{f4}{e3};
  \move{g4}{f4};
  \move{h5}{g4};
\end{struggleboard}
\end{center}

\newpage

\subsection*{Teleporter}

Choose a piece to be your teleporter (traditionally blue).
This piece can move regularly, or it can use the power of teleportation.
If the power is used, the teleporter may move to any open space not behind it within three spaces of itself.
When the power is used, the pieces over which the teleporter moves are not considered jumped.

Below, the blue piece is our teleporter, and all spaces he is able to move to are highlighted. The jumps drawn below indicate that the teleporter can also move regularly to these spots using the standard movement rules for jumping. A space is highlighted green if the piece can move there by means of teleporter, purple if the piece can move there by teleporting, and red if the piece can only reach the spot by moving regularly.

\

\begin{center}
\begin{struggleboard}
  \piece{h5}{blue};
  \highlight{h6}{purple};
  \highlight{h4}{purple};
  \highlight{h3}{green};
  \highlight{h2}{green};
  \highlight{g2}{green};
  \highlight{g3}{green};
  \highlight{g4}{purple};
  \highlight{g5}{purple};
  \highlight{f3}{green};
  \highlight{f4}{green};
  \highlight{f5}{green};
  \highlight{f6}{green};
  \highlight{e3}{green};
  \highlight{e4}{green};
  \highlight{e5}{green};
\end{struggleboard}
\begin{struggleboard}
  \piece{i2}{blue};
  \piece{h3}{yellow};
  \piece{f4}{green};
  \piece{e3}{black};
  \highlight{i1}{purple};
  \highlight{i3}{purple};
  \highlight{g3}{purple};
  \highlight{h2}{purple};
  \highlight{i4}{green};
  \highlight{i5}{green};
  \highlight{h1}{green};
  \highlight{h4}{green};
  \highlight{h5}{green};
  \highlight{g1}{green};
  \highlight{g2}{green};
  \highlight{g4}{green};
  \highlight{f1}{green};
  \highlight{f2}{green};
  \highlight{f3}{green};
  \highlight{e4}{red};
  \highlight{e2}{red};
\end{struggleboard}
\end{center}


\end{document}