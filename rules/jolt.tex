\documentclass[../rulebook.tex]{subfiles}

\begin{document}
\subsection*{Jolt}

Designate one piece to be your Jolt (traditionally yellow).
When this piece jumps over any of your nonwhite pieces (`jolts' them,
you can move the jumped-over piece using a standard move
(that is, the piece may move normally except for that it is not
allowed to use its power(s). If you jolt more than one piece,
you may choose the order in which you move them.

Below, the yellow is your jolt, the blue is your teleporter,
and the red is also your piece.
The green and the black belong to your opponent.

\begin{center}
  \begin{struggleboard}
    \piece{j3}{yellow};
    \piece{i3}{red};
    \piece{h4}{blue};
    \piece{g5}{black};
    \piece{g3}{green};
    \move{j3}{h3};
    \move{h3}{h5};
    \move{h5}{f6};
  \end{struggleboard}
  \begin{struggleboard}
    \piece{f6}{yellow};
    \piece{i3}{red};
    \piece{h4}{blue};
    \piece{g3}{green};
    \move{i3}{g4};
  \end{struggleboard}
\end{center}
\
\begin{center}
  \begin{struggleboard}
    \piece{f6}{yellow};
    \piece{g4}{red};
    \piece{h4}{blue};
    \piece{g3}{green};
    \move{h4}{f3};
  \end{struggleboard}
  \begin{struggleboard}
    \piece{f6}{yellow};
    \piece{i3}{red};
    \piece{f3}{blue};
  \end{struggleboard}
\end{center}

\

In the above, your jolt jumps your red, your blue,
and your opponent's black. Then, since you jolted your
red and your blue, you can move either or both of them.
Here, you move your red first, and then you move your blue.

\end{document}
