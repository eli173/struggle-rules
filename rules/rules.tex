\documentclass[../rulebook.tex]{subfiles}

\begin{document}
\section*{Struggle (Working Title)}
\subsection*{Overview and Objective}
Each player has a team of 6 pieces of 6 different colors:
white, black, blue, green, red, and yellow.
Pieces move by moving to adjacent spaces or by jumping over other pieces;
enemy pieces are captured when they are jumped.
A player wins in one of two circumstances:
(1) he captures his opponent’s white,
or (2) he moves his white across the board to his opponent’s
white’s starting position. 

Before play, each player chooses three different Rules
from the Rules list.
These Powers are used throughout the game to extend the abilities
of single pieces or of the whole team. 

\subsection*{Board and Setup}
Struggle is played on a modified Chinese checkers board in which 4 of the 6 home-base triangles have been sectioned off, leaving a diamond-shaped Struggle board.  The first “layer” of the inner diamond and the last spaces on either side of the middle row are also sectioned off, so that the diamond’s sides are 6 spaces long.


Players set up pieces at either end of the diamond.
White is placed in the first space facing the player
(a row containing only one space),
and the other five pieces are placed in the next two rows
as the player wishes. (In competitive games, players may want to secretly
decide on their starting positions and reveal them simultaneously.)

\subsection*{Movement and Capture}
Players make 1 move per turn.
Non-Rule (conventional) movement consists of either
(1) moving to an adjacent, unoccupied space or
(2) jumping over an adjacent piece with an unoccupied
space directly behind it, in the direction of the jump
(as in checkers or Chinese checkers).
A player may jump both his own and his opponent’s pieces:
when an opponent’s piece is jumped, it is removed from the board
unless a Power which says otherwise is invoked.

Conventional movement cannot be backwards:
pieces moving conventionally must step or jump either forward
(diagonally) or horizontally.
After performing a conventional jump,
a piece may continue to jump over jumpable pieces.
A piece which is allowed to jump adjacent pieces after a jump
is said to have `bounce'.
A piece which has just performed a conventional movement
is said to be landing.

\subsection*{The White Piece and Zapping}
The game ends when either (1) one player’s white piece is captured or
(2) one player’s white piece lands in the opponent’s
white’s starting position.

When a player’s white lands on the fourth row from his opponent
(the 10th row from his perspective),
any pieces in the opponent’s home base
(the rows in which his pieces began) are immediately captured:
they are said to be `zapped'.
This is to guard against camping of the home base.

\subsection*{General Guidelines for Rules}
Rules are intended to make the game more fun, interesting, and deep.
Players encouraged to develop their own Rules, sets of Rules,
and strategies based around them.
We have used the following guidelines so far when coming up with Rules:
 Avoid making Rules that affect white
 Ban the stacking of Rules which are broken in combination
 Avoid making Rules that tend to stall the game
 Shoot for rules that are both fun to play with and against
\end{document}
