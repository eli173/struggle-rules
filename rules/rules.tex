\documentclass[../rulebook.tex]{subfiles}

\begin{document}
\section*{Struggle (Working Title)}
\subsection*{Overview and Objective}
Each player has a team of 6 pieces of 6 different colors:
white, black, blue, green, red, and yellow.
Pieces generally move by moving to adjacent spaces or by jumping over other pieces;
enemy pieces are captured when they are jumped.
A player wins in one of two circumstances:
\begin{itemize}\itemsep0pt
\item he captures his opponent's white, or 
\item he moves his white across the board to his opponent's white's starting position
\end{itemize}

Before the game begins, each player chooses three different Rules
from the Rule list.
These Rules are used throughout the game to extend the abilities
of single pieces or of the whole team.

\subsection*{Board and Setup}
Struggle is played on a modified Chinese checkers board in which 4 of the 6 home-base triangles have been sectioned off, leaving a diamond-shaped Struggle board.  The first “layer” of the inner diamond and the last spaces on either side of the middle row are also sectioned off, so that the diamond's sides are 6 spaces long.


Players set up pieces at either end of the diamond.
White is placed in the first space facing the player
(a row containing only one space),
and the other five pieces are placed in the next two rows
as the player wishes. (In competitive games, players may want to secretly
decide on their starting positions and reveal them simultaneously.)

An example starting board is pictured below.

\begin{center}
\begin{struggleboard}
% p1
  \piece{m1}{white};
  \piece{l1}{blue};
  \piece{l2}{red};
  \piece{k1}{green};
  \piece{k2}{yellow};
  \piece{k3}{black};
% p2
  \piece{a1}{white};
  \piece{b1}{black};
  \piece{b2}{green};
  \piece{c1}{yellow};
  \piece{c2}{blue};
  \piece{c3}{red};
\end{struggleboard}
\end{center}

Players play in turns, and players can either decide between themselves
who should go first or play rock-paper-scissors to decide.

A concept used in some special rules (detailed later in this document)
is that of a `cycle'.
A cycle is a length of time consisting of three turns,
and here a turn includes both you and your opponent's move.
For example, if you invoke a rule at the end of your turn that lasts
for one cycle, the move will last as your opponent moves, then you move,
then your opponent, you, opponent, you, and finally the cycle is over so
the rule no longer has its effect.



You also have the option to use a chess clock when you play.
This can help keep games moving and make sure games don't last too long.
Typically, the clocks are set to ten minutes per player,
but feel free to use as much or as little time as you would like.


\subsection*{Movement and Capture}
Players make 1 move per turn.
Non-Rule (conventional) movement consists of either
(1) moving to an adjacent, unoccupied space or
(2) jumping over an adjacent piece with an unoccupied
space directly behind it, in the direction of the jump
(as in checkers or Chinese checkers).
A player may jump both his own and his opponent's pieces:
when an opponent's piece is jumped, it is removed from the board
unless a Power which says otherwise is invoked.

Conventional movement cannot be backwards:
pieces moving conventionally must step or jump either forward
(diagonally) or horizontally.
After performing a conventional jump,
a piece may continue to jump over jumpable pieces.
A piece which is allowed to jump adjacent pieces after a jump
is said to have `bounce'.
A piece which has just performed a conventional movement
is said to be `landing'.

\subsection*{The White Piece and Zapping}
The game ends when either (1) one player's white piece is captured or
(2) one player's white piece lands in the opponent's
white's starting position.

When a player's white lands on the fourth row from his opponent
(the 10th row from his perspective),
any pieces in the opponent's home base
(the rows in which his pieces began) are immediately captured:
they are said to be `zapped'.
This is to guard against camping of the home base.

\subsection*{Special Rules}
On the following pages, there is a list of special rules used
to enhance the game.
Before the game, you and your opponent choose three rules from the 
following list. 
These rules give the game variety; with just a handful of rules
you get hundreds of combinations, so you could play a different game
each time.
The rules vary in scope and effect: some apply every turn,
some are only used once or twice a game,
some grant greater offensive capabilities,
others improve your defense,
and others are used to trap your opponent or trick him or her
into a dangerous situation.
When choosing rules, think about how the rules play
with and against each other, and how certain rules might or might
not fit your style or give you the expressiveness in play you want.

Rules are intended to make the game more fun, interesting, and deep.
Players are encouraged to develop their own Rules, sets of Rules,
and strategies based around them.
We have found that using the following guidelines to help us come up
with rules that make the game more interesting and keep it fairly balanced:
\begin{itemize}
\item Avoid making Rules that affect white
\item Ban the stacking of Rules which are broken in combination
\item Avoid making Rules that tend to stall the game excessively
\item Shoot for rules that are both fun to play with and against
\end{itemize}
Feel free to follow all, some, or even none of these guidelines.
The most important thing about designing rules is making sure
that the rule makes the game more fun or interesting for you to play.

\end{document}
